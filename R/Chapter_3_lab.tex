\documentclass[fleqn, a4paper]{article}
\usepackage{kotex, amssymb}
\usepackage{Sweave}
\begin{document}
\Sconcordance{concordance:Chapter_3_lab.tex:Chapter_3_lab.Rnw:%
1 2 1 1 0 72 1}


\begin{enumerate}
\item[3-5.] 3장 5번 문제와 추가 문제 
\begin{enumerate}
\item[Q1.] $X^tX = X\Rightarrow X = X^t = X^2$.
\item[(풀이 1)] $A^tA = 0$ 이면 $A = 0$인 성질을 이용. $A=X-X^2$라 하면,
\begin{eqnarray*}
A^tA&=&(X-X^2)^t(X-X^2)\\
      &=&X^tX-(X^2)^tX - X^tX^2 + (X^2)^t(X^2)\\
      &=&X^tX-X^tX^tX-X^tX\ X + X^tX^tX\ X\\
      &=&X^tX-X^t(X^tX)-(X^tX)\ X + X^t(X^tX)X\\
      &=&X^tX-X^tX-X\ X + (X^tX)\ X\\
      &=&X^tX-X^tX-X\ X + X\ X = 0\\
      \therefore X-X^2&=&0\\
      \Rightarrow X&=&X^2 
\end{eqnarray*}
\item[(풀이 2)] A1의 결과를 이용. 이번에는 $A=X-X^t$라 하면,
\begin{eqnarray*}
A^tA &=&(X-X^t)^t(X-X^t)\\ 
      &=&X^tX-(X^t)^tX - X^tX^t + (X^t)^tX^t\\
      &=&(X^tX)-X\ X - (X\ X)^t + X\ X^t\\
      &=&X-X^2-(X^2)^t + (X^tX)^t\\
      &=&0-X^t + X^t = 0\\
      \therefore X-X^t&=&0\\
      \Rightarrow X&=&X^t 
\end{eqnarray*}
\item[Q2.] $X^tX = X^2\Rightarrow X = X^t$.
\item[(풀이)] ${\rm tr}(A^tA)=0$이면 $A=0$인 성질을 이용. $A=X-X^t$라 하면,
\begin{eqnarray*}
{\rm tr}(A^tA)&=&{\rm tr}[(X-X^t)^t(X-X^t)]\\ 
      &=&{\rm tr}[X^tX-(X^t)^tX - X^tX^t + (X^t)^tX^t]\\
      &=&{\rm tr}(X^tX)-{\rm tr}(X\ X) - {\rm tr}[(X\ X)^t] + {\rm tr}[X\ X^t]\\
      &=&{\rm tr}(X^2)-{\rm tr}(X^2)-{\rm tr}[(X^2)^t] + {\rm tr}[(X^tX)^t]\\
      &=&0-{\rm tr}[(X^2)^t] + {\rm tr}[(X^2)^t]\\
      &=&0\\
      \therefore A&=&X-X^t=0\\
      \Rightarrow X&=&X^t 
\end{eqnarray*}
\end{enumerate}
\newcommand{\bm}[1]{\mbox{\boldmath{$#1$}}}
\item[1-7.] 두 가지 방법으로 설명.
\begin{enumerate}
\item[(풀이 1)] 합 기호를 이용하여 보이면,
\begin{eqnarray*}
\sum_{i=1}^m\mu_ix_i&=&\sum_{i=1}^m\mu_i\sum_{j=1}^n\lambda_{ij}v_j\\
&=&\sum_{i=1}^m\sum_{j=1}^n\mu_i\lambda_{ij}v_j\\
&=&\sum_{j=1}^n\sum_{i=1}^m\mu_i\lambda_{ij}v_j\\
&=&\sum_{j=1}^n\big(\sum_{i=1}^m\mu_i\lambda_{ij}\big)v_j\\
\end{eqnarray*}
로부터 $v_j$의 계수들의 합은
\begin{eqnarray*}
\sum_{j=1}^n\big(\sum_{i=1}^m\mu_i\lambda_{ij}\big)&=&\sum_{i=1}^m\sum_{j=1}^n\mu_i\lambda_{ij}\\
&=&\sum_{i=1}^m\mu_i\big(\sum_{j=1}^n\lambda_{ij}\big)\\
&=&\sum_{i=1}^m\mu_i\cdot1\\
&=&1
\end{eqnarray*}
\item[(풀이 2)] 행렬과 summing vector $\bm{1}$ 을 이용하여 보이면,

$\bm{x} =(x_1, \ldots, x_m)^t$, $\Lambda=\{\lambda_{ij}\}$, $\bm{v}=(v_1,\ldots, v_n)^t$라 할 때, 주어진 조건은 $\bm{x}=\Lambda\ \bm{v}$, $\Lambda\ \bm{1}_n = \bm{1}_m$, $\bm{\mu}^t\bm{1}_m=1$이라고 정리할 수 있음. 
\begin{eqnarray*}
\sum_{i=1}^m\mu_ix_i=\bm{\mu}^t\bm{x}=\bm{\mu}^t\Lambda\ \bm{v}
\end{eqnarray*}
로부터 $\sum_{i=1}^m\mu_ix_i$에서 $v_j$의 계수들의 합은 
 $\bm{\mu}^t\Lambda\ \bm{1}_n$이므로
\begin{eqnarray*}
\bm{\mu}^t\Lambda\ \bm{1}_n&=&\bm{\mu}^t\bm{1}_m\\
&=&1
\end{eqnarray*}
\end{enumerate}
\end{enumerate}
\end{document}
